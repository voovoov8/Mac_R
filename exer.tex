% Options for packages loaded elsewhere
\PassOptionsToPackage{unicode}{hyperref}
\PassOptionsToPackage{hyphens}{url}
\PassOptionsToPackage{dvipsnames,svgnames,x11names}{xcolor}
%
\documentclass[
  letterpaper,
  DIV=11,
  numbers=noendperiod]{scrartcl}

\usepackage{amsmath,amssymb}
\usepackage{iftex}
\ifPDFTeX
  \usepackage[T1]{fontenc}
  \usepackage[utf8]{inputenc}
  \usepackage{textcomp} % provide euro and other symbols
\else % if luatex or xetex
  \usepackage{unicode-math}
  \defaultfontfeatures{Scale=MatchLowercase}
  \defaultfontfeatures[\rmfamily]{Ligatures=TeX,Scale=1}
\fi
\usepackage{lmodern}
\ifPDFTeX\else  
    % xetex/luatex font selection
\fi
% Use upquote if available, for straight quotes in verbatim environments
\IfFileExists{upquote.sty}{\usepackage{upquote}}{}
\IfFileExists{microtype.sty}{% use microtype if available
  \usepackage[]{microtype}
  \UseMicrotypeSet[protrusion]{basicmath} % disable protrusion for tt fonts
}{}
\makeatletter
\@ifundefined{KOMAClassName}{% if non-KOMA class
  \IfFileExists{parskip.sty}{%
    \usepackage{parskip}
  }{% else
    \setlength{\parindent}{0pt}
    \setlength{\parskip}{6pt plus 2pt minus 1pt}}
}{% if KOMA class
  \KOMAoptions{parskip=half}}
\makeatother
\usepackage{xcolor}
\setlength{\emergencystretch}{3em} % prevent overfull lines
\setcounter{secnumdepth}{-\maxdimen} % remove section numbering
% Make \paragraph and \subparagraph free-standing
\makeatletter
\ifx\paragraph\undefined\else
  \let\oldparagraph\paragraph
  \renewcommand{\paragraph}{
    \@ifstar
      \xxxParagraphStar
      \xxxParagraphNoStar
  }
  \newcommand{\xxxParagraphStar}[1]{\oldparagraph*{#1}\mbox{}}
  \newcommand{\xxxParagraphNoStar}[1]{\oldparagraph{#1}\mbox{}}
\fi
\ifx\subparagraph\undefined\else
  \let\oldsubparagraph\subparagraph
  \renewcommand{\subparagraph}{
    \@ifstar
      \xxxSubParagraphStar
      \xxxSubParagraphNoStar
  }
  \newcommand{\xxxSubParagraphStar}[1]{\oldsubparagraph*{#1}\mbox{}}
  \newcommand{\xxxSubParagraphNoStar}[1]{\oldsubparagraph{#1}\mbox{}}
\fi
\makeatother

\usepackage{color}
\usepackage{fancyvrb}
\newcommand{\VerbBar}{|}
\newcommand{\VERB}{\Verb[commandchars=\\\{\}]}
\DefineVerbatimEnvironment{Highlighting}{Verbatim}{commandchars=\\\{\}}
% Add ',fontsize=\small' for more characters per line
\usepackage{framed}
\definecolor{shadecolor}{RGB}{241,243,245}
\newenvironment{Shaded}{\begin{snugshade}}{\end{snugshade}}
\newcommand{\AlertTok}[1]{\textcolor[rgb]{0.68,0.00,0.00}{#1}}
\newcommand{\AnnotationTok}[1]{\textcolor[rgb]{0.37,0.37,0.37}{#1}}
\newcommand{\AttributeTok}[1]{\textcolor[rgb]{0.40,0.45,0.13}{#1}}
\newcommand{\BaseNTok}[1]{\textcolor[rgb]{0.68,0.00,0.00}{#1}}
\newcommand{\BuiltInTok}[1]{\textcolor[rgb]{0.00,0.23,0.31}{#1}}
\newcommand{\CharTok}[1]{\textcolor[rgb]{0.13,0.47,0.30}{#1}}
\newcommand{\CommentTok}[1]{\textcolor[rgb]{0.37,0.37,0.37}{#1}}
\newcommand{\CommentVarTok}[1]{\textcolor[rgb]{0.37,0.37,0.37}{\textit{#1}}}
\newcommand{\ConstantTok}[1]{\textcolor[rgb]{0.56,0.35,0.01}{#1}}
\newcommand{\ControlFlowTok}[1]{\textcolor[rgb]{0.00,0.23,0.31}{\textbf{#1}}}
\newcommand{\DataTypeTok}[1]{\textcolor[rgb]{0.68,0.00,0.00}{#1}}
\newcommand{\DecValTok}[1]{\textcolor[rgb]{0.68,0.00,0.00}{#1}}
\newcommand{\DocumentationTok}[1]{\textcolor[rgb]{0.37,0.37,0.37}{\textit{#1}}}
\newcommand{\ErrorTok}[1]{\textcolor[rgb]{0.68,0.00,0.00}{#1}}
\newcommand{\ExtensionTok}[1]{\textcolor[rgb]{0.00,0.23,0.31}{#1}}
\newcommand{\FloatTok}[1]{\textcolor[rgb]{0.68,0.00,0.00}{#1}}
\newcommand{\FunctionTok}[1]{\textcolor[rgb]{0.28,0.35,0.67}{#1}}
\newcommand{\ImportTok}[1]{\textcolor[rgb]{0.00,0.46,0.62}{#1}}
\newcommand{\InformationTok}[1]{\textcolor[rgb]{0.37,0.37,0.37}{#1}}
\newcommand{\KeywordTok}[1]{\textcolor[rgb]{0.00,0.23,0.31}{\textbf{#1}}}
\newcommand{\NormalTok}[1]{\textcolor[rgb]{0.00,0.23,0.31}{#1}}
\newcommand{\OperatorTok}[1]{\textcolor[rgb]{0.37,0.37,0.37}{#1}}
\newcommand{\OtherTok}[1]{\textcolor[rgb]{0.00,0.23,0.31}{#1}}
\newcommand{\PreprocessorTok}[1]{\textcolor[rgb]{0.68,0.00,0.00}{#1}}
\newcommand{\RegionMarkerTok}[1]{\textcolor[rgb]{0.00,0.23,0.31}{#1}}
\newcommand{\SpecialCharTok}[1]{\textcolor[rgb]{0.37,0.37,0.37}{#1}}
\newcommand{\SpecialStringTok}[1]{\textcolor[rgb]{0.13,0.47,0.30}{#1}}
\newcommand{\StringTok}[1]{\textcolor[rgb]{0.13,0.47,0.30}{#1}}
\newcommand{\VariableTok}[1]{\textcolor[rgb]{0.07,0.07,0.07}{#1}}
\newcommand{\VerbatimStringTok}[1]{\textcolor[rgb]{0.13,0.47,0.30}{#1}}
\newcommand{\WarningTok}[1]{\textcolor[rgb]{0.37,0.37,0.37}{\textit{#1}}}

\providecommand{\tightlist}{%
  \setlength{\itemsep}{0pt}\setlength{\parskip}{0pt}}\usepackage{longtable,booktabs,array}
\usepackage{calc} % for calculating minipage widths
% Correct order of tables after \paragraph or \subparagraph
\usepackage{etoolbox}
\makeatletter
\patchcmd\longtable{\par}{\if@noskipsec\mbox{}\fi\par}{}{}
\makeatother
% Allow footnotes in longtable head/foot
\IfFileExists{footnotehyper.sty}{\usepackage{footnotehyper}}{\usepackage{footnote}}
\makesavenoteenv{longtable}
\usepackage{graphicx}
\makeatletter
\newsavebox\pandoc@box
\newcommand*\pandocbounded[1]{% scales image to fit in text height/width
  \sbox\pandoc@box{#1}%
  \Gscale@div\@tempa{\textheight}{\dimexpr\ht\pandoc@box+\dp\pandoc@box\relax}%
  \Gscale@div\@tempb{\linewidth}{\wd\pandoc@box}%
  \ifdim\@tempb\p@<\@tempa\p@\let\@tempa\@tempb\fi% select the smaller of both
  \ifdim\@tempa\p@<\p@\scalebox{\@tempa}{\usebox\pandoc@box}%
  \else\usebox{\pandoc@box}%
  \fi%
}
% Set default figure placement to htbp
\def\fps@figure{htbp}
\makeatother

\KOMAoption{captions}{tableheading}
\makeatletter
\@ifpackageloaded{caption}{}{\usepackage{caption}}
\AtBeginDocument{%
\ifdefined\contentsname
  \renewcommand*\contentsname{Table of contents}
\else
  \newcommand\contentsname{Table of contents}
\fi
\ifdefined\listfigurename
  \renewcommand*\listfigurename{List of Figures}
\else
  \newcommand\listfigurename{List of Figures}
\fi
\ifdefined\listtablename
  \renewcommand*\listtablename{List of Tables}
\else
  \newcommand\listtablename{List of Tables}
\fi
\ifdefined\figurename
  \renewcommand*\figurename{Figure}
\else
  \newcommand\figurename{Figure}
\fi
\ifdefined\tablename
  \renewcommand*\tablename{Table}
\else
  \newcommand\tablename{Table}
\fi
}
\@ifpackageloaded{float}{}{\usepackage{float}}
\floatstyle{ruled}
\@ifundefined{c@chapter}{\newfloat{codelisting}{h}{lop}}{\newfloat{codelisting}{h}{lop}[chapter]}
\floatname{codelisting}{Listing}
\newcommand*\listoflistings{\listof{codelisting}{List of Listings}}
\makeatother
\makeatletter
\makeatother
\makeatletter
\@ifpackageloaded{caption}{}{\usepackage{caption}}
\@ifpackageloaded{subcaption}{}{\usepackage{subcaption}}
\makeatother

\usepackage{bookmark}

\IfFileExists{xurl.sty}{\usepackage{xurl}}{} % add URL line breaks if available
\urlstyle{same} % disable monospaced font for URLs
\hypersetup{
  pdftitle={exer},
  pdfauthor={yoon},
  colorlinks=true,
  linkcolor={blue},
  filecolor={Maroon},
  citecolor={Blue},
  urlcolor={Blue},
  pdfcreator={LaTeX via pandoc}}


\title{exer}
\author{yoon}
\date{}

\begin{document}
\maketitle


\section{기말 연습용}\label{uxae30uxb9d0-uxc5f0uxc2b5uxc6a9}

\subsection{1. 선형을 활용한
예측}\label{uxc120uxd615uxc744-uxd65cuxc6a9uxd55c-uxc608uxce21}

\subsubsection{1-1. loading}\label{loading}

\begin{Shaded}
\begin{Highlighting}[]
\FunctionTok{library}\NormalTok{(caret)}
\end{Highlighting}
\end{Shaded}

\begin{verbatim}
Loading required package: ggplot2
\end{verbatim}

\begin{verbatim}
Loading required package: lattice
\end{verbatim}

\begin{Shaded}
\begin{Highlighting}[]
\FunctionTok{library}\NormalTok{(MASS)}
\FunctionTok{library}\NormalTok{(tidyverse)}
\end{Highlighting}
\end{Shaded}

\begin{verbatim}
-- Attaching core tidyverse packages ------------------------ tidyverse 2.0.0 --
v dplyr     1.1.4     v readr     2.1.4
v forcats   1.0.0     v stringr   1.5.1
v lubridate 1.9.3     v tibble    3.2.1
v purrr     1.0.2     v tidyr     1.3.0
\end{verbatim}

\begin{verbatim}
-- Conflicts ------------------------------------------ tidyverse_conflicts() --
x dplyr::filter() masks stats::filter()
x dplyr::lag()    masks stats::lag()
x purrr::lift()   masks caret::lift()
x dplyr::select() masks MASS::select()
i Use the conflicted package (<http://conflicted.r-lib.org/>) to force all conflicts to become errors
\end{verbatim}

\begin{Shaded}
\begin{Highlighting}[]
\NormalTok{gpa }\OtherTok{\textless{}{-}} \FunctionTok{read.csv}\NormalTok{(}\StringTok{"dataforclass/gpa1.csv"}\NormalTok{,}\AttributeTok{header =} \ConstantTok{TRUE}\NormalTok{,}
                \AttributeTok{sep =} \StringTok{"}\SpecialCharTok{\textbackslash{}t}\StringTok{"}\NormalTok{)}
\FunctionTok{str}\NormalTok{(gpa)}
\end{Highlighting}
\end{Shaded}

\begin{verbatim}
'data.frame':   141 obs. of  29 variables:
 $ age     : int  21 21 20 19 20 20 22 22 22 19 ...
 $ soph    : int  0 0 0 1 0 0 0 0 0 1 ...
 $ junior  : int  0 0 1 0 1 0 0 0 0 0 ...
 $ senior  : int  1 1 0 0 0 1 0 0 0 0 ...
 $ senior5 : int  0 0 0 0 0 0 1 1 1 0 ...
 $ male    : int  0 0 0 1 0 1 0 0 0 0 ...
 $ campus  : int  0 0 0 1 0 1 0 0 0 0 ...
 $ business: int  1 1 1 1 1 1 1 0 0 1 ...
 $ engineer: int  0 0 0 0 0 0 0 0 0 0 ...
 $ colGPA  : num  3 3.4 3 3.5 3.6 ...
 $ hsGPA   : num  3 3.2 3.6 3.5 3.9 ...
 $ ACT     : int  21 24 26 27 28 25 25 22 21 27 ...
 $ job19   : int  0 0 1 1 0 0 0 1 1 1 ...
 $ job20   : int  1 1 0 0 1 0 0 0 0 0 ...
 $ drive   : int  1 1 0 0 0 0 0 1 1 0 ...
 $ bike    : int  0 0 0 0 1 0 1 0 0 0 ...
 $ walk    : int  0 0 1 1 0 1 0 0 0 1 ...
 $ voluntr : int  0 0 0 0 0 0 0 0 0 0 ...
 $ PC      : int  0 0 0 0 0 0 0 1 0 1 ...
 $ greek   : int  0 0 0 0 0 0 1 0 0 0 ...
 $ car     : int  1 1 1 0 1 1 1 0 1 0 ...
 $ siblings: int  1 0 1 1 1 1 1 1 1 1 ...
 $ bgfriend: int  0 1 0 0 1 0 0 1 1 0 ...
 $ clubs   : int  0 1 1 0 0 0 1 0 1 1 ...
 $ skipped : num  2 0 0 0 0 0 0 3 2 0.5 ...
 $ alcohol : num  1 1 1 0 1.5 0 2 3 2.5 0.75 ...
 $ gradMI  : int  1 1 1 0 1 0 1 1 1 1 ...
 $ fathcoll: int  0 1 1 0 1 1 0 1 1 0 ...
 $ mothcoll: int  0 1 1 0 0 0 1 1 1 1 ...
\end{verbatim}

\subsubsection{1-2. train과 test 데이터
구분}\label{trainuxacfc-test-uxb370uxc774uxd130-uxad6cuxbd84}

\begin{itemize}
\tightlist
\item
  데이터 구분
\end{itemize}

\begin{Shaded}
\begin{Highlighting}[]
\NormalTok{train\_i }\OtherTok{\textless{}{-}} \FunctionTok{createDataPartition}\NormalTok{(gpa}\SpecialCharTok{$}\NormalTok{colGPA,}
                               \AttributeTok{p =} \FloatTok{0.8}\NormalTok{,}
                               \AttributeTok{list =} \ConstantTok{FALSE}\NormalTok{) }
\NormalTok{train }\OtherTok{\textless{}{-}}\NormalTok{ gpa[train\_i,]}
\NormalTok{test }\OtherTok{\textless{}{-}}\NormalTok{ gpa[}\SpecialCharTok{{-}}\NormalTok{train\_i,]}
\FunctionTok{str}\NormalTok{(train)}
\end{Highlighting}
\end{Shaded}

\begin{verbatim}
'data.frame':   115 obs. of  29 variables:
 $ age     : int  21 20 19 20 20 22 22 21 22 21 ...
 $ soph    : int  0 0 1 0 0 0 0 0 0 0 ...
 $ junior  : int  0 1 0 1 0 0 0 0 0 0 ...
 $ senior  : int  1 0 0 0 1 0 0 1 0 1 ...
 $ senior5 : int  0 0 0 0 0 1 1 0 1 0 ...
 $ male    : int  0 0 1 0 1 0 0 1 0 0 ...
 $ campus  : int  0 0 1 0 1 0 0 0 1 1 ...
 $ business: int  1 1 1 1 1 1 0 1 1 1 ...
 $ engineer: int  0 0 0 0 0 0 0 0 0 0 ...
 $ colGPA  : num  3.4 3 3.5 3.6 3 ...
 $ hsGPA   : num  3.2 3.6 3.5 3.9 3.4 ...
 $ ACT     : int  24 26 27 28 25 25 22 19 22 23 ...
 $ job19   : int  0 1 1 0 0 0 1 1 0 1 ...
 $ job20   : int  1 0 0 1 0 0 0 0 1 0 ...
 $ drive   : int  1 0 0 0 0 0 1 0 0 0 ...
 $ bike    : int  0 0 0 1 0 1 0 0 0 0 ...
 $ walk    : int  0 1 1 0 1 0 0 1 1 1 ...
 $ voluntr : int  0 0 0 0 0 0 0 0 1 0 ...
 $ PC      : int  0 0 0 0 0 0 1 0 0 1 ...
 $ greek   : int  0 0 0 0 0 1 0 0 0 0 ...
 $ car     : int  1 1 0 1 1 1 0 1 1 1 ...
 $ siblings: int  0 1 1 1 1 1 1 1 1 1 ...
 $ bgfriend: int  1 0 0 1 0 0 1 1 1 1 ...
 $ clubs   : int  1 1 0 0 0 1 0 0 1 1 ...
 $ skipped : num  0 0 0 0 0 0 3 2 1 0 ...
 $ alcohol : num  1 1 0 1.5 0 2 3 1 1 1 ...
 $ gradMI  : int  1 1 0 1 0 1 1 1 1 1 ...
 $ fathcoll: int  1 1 0 1 1 0 1 0 0 0 ...
 $ mothcoll: int  1 1 0 0 0 1 1 0 1 1 ...
\end{verbatim}

\begin{Shaded}
\begin{Highlighting}[]
\FunctionTok{str}\NormalTok{(test)}
\end{Highlighting}
\end{Shaded}

\begin{verbatim}
'data.frame':   26 obs. of  29 variables:
 $ age     : int  21 22 19 20 22 22 20 20 22 21 ...
 $ soph    : int  0 0 1 0 0 0 0 0 0 0 ...
 $ junior  : int  0 0 0 1 0 0 1 1 0 0 ...
 $ senior  : int  1 0 0 0 1 1 0 0 1 1 ...
 $ senior5 : int  0 1 0 0 0 0 0 0 0 0 ...
 $ male    : int  0 0 0 1 1 1 0 0 1 1 ...
 $ campus  : int  0 0 0 0 0 0 0 0 0 0 ...
 $ business: int  1 0 1 1 1 0 1 1 1 1 ...
 $ engineer: int  0 0 0 0 0 1 0 0 0 0 ...
 $ colGPA  : num  3 2.7 3.8 3.3 2.6 ...
 $ hsGPA   : num  3 3 4 3.7 3 ...
 $ ACT     : int  21 21 27 25 21 23 24 27 22 28 ...
 $ job19   : int  0 1 1 0 0 1 1 1 0 0 ...
 $ job20   : int  1 0 0 0 0 0 0 0 0 1 ...
 $ drive   : int  1 1 0 1 1 0 0 0 0 0 ...
 $ bike    : int  0 0 0 0 0 0 0 1 1 1 ...
 $ walk    : int  0 0 1 0 0 1 1 0 0 0 ...
 $ voluntr : int  0 0 0 0 0 0 1 0 0 0 ...
 $ PC      : int  0 0 1 1 0 0 0 1 1 0 ...
 $ greek   : int  0 0 0 1 0 0 0 1 0 0 ...
 $ car     : int  1 1 0 0 1 1 1 0 1 1 ...
 $ siblings: int  1 1 1 1 1 1 1 1 1 1 ...
 $ bgfriend: int  0 1 0 0 0 0 1 1 1 1 ...
 $ clubs   : int  0 1 1 0 1 1 1 0 1 1 ...
 $ skipped : num  2 2 0.5 1 3 1 1 2 0 1 ...
 $ alcohol : num  1 2.5 0.75 4 3.5 1 1 1 1 3 ...
 $ gradMI  : int  1 1 1 1 1 1 1 1 1 1 ...
 $ fathcoll: int  0 1 0 1 1 0 1 0 1 1 ...
 $ mothcoll: int  0 1 1 1 0 1 0 0 0 1 ...
\end{verbatim}

\subsubsection{1-3. 선형모형 제작과 예측
실험}\label{uxc120uxd615uxbaa8uxd615-uxc81cuxc791uxacfc-uxc608uxce21-uxc2e4uxd5d8}

\begin{itemize}
\tightlist
\item
  선형 모델 제작
\end{itemize}

\begin{Shaded}
\begin{Highlighting}[]
\NormalTok{lm\_gpa }\OtherTok{\textless{}{-}} \FunctionTok{lm}\NormalTok{(colGPA}\SpecialCharTok{\textasciitilde{}}\NormalTok{., }\AttributeTok{data =}\NormalTok{ train)}
\FunctionTok{summary}\NormalTok{(lm\_gpa)}
\end{Highlighting}
\end{Shaded}

\begin{verbatim}

Call:
lm(formula = colGPA ~ ., data = train)

Residuals:
     Min       1Q   Median       3Q      Max 
-0.69857 -0.19454  0.04212  0.22320  0.55829 

Coefficients: (2 not defined because of singularities)
             Estimate Std. Error t value Pr(>|t|)    
(Intercept)  0.439129   0.906852   0.484 0.629423    
age          0.027613   0.033851   0.816 0.416850    
soph         0.382731   0.322543   1.187 0.238578    
junior       0.038615   0.139489   0.277 0.782557    
senior       0.021879   0.123219   0.178 0.859474    
senior5            NA         NA      NA       NA    
male         0.067568   0.081609   0.828 0.409942    
campus      -0.093568   0.092306  -1.014 0.313516    
business     0.090142   0.101154   0.891 0.375290    
engineer    -0.399610   0.250107  -1.598 0.113682    
hsGPA        0.460066   0.119641   3.845 0.000227 ***
ACT          0.013706   0.013451   1.019 0.311008    
job19       -0.019026   0.074880  -0.254 0.800018    
job20       -0.088617   0.098254  -0.902 0.369560    
drive        0.066776   0.106076   0.630 0.530645    
bike        -0.011698   0.080896  -0.145 0.885353    
walk               NA         NA      NA       NA    
voluntr     -0.076004   0.085487  -0.889 0.376388    
PC           0.114023   0.074768   1.525 0.130841    
greek        0.054252   0.075073   0.723 0.471809    
car         -0.122867   0.087304  -1.407 0.162846    
siblings    -0.133382   0.136118  -0.980 0.329821    
bgfriend     0.108152   0.067313   1.607 0.111703    
clubs        0.087596   0.069346   1.263 0.209859    
skipped     -0.085825   0.034938  -2.456 0.015993 *  
alcohol      0.004714   0.030994   0.152 0.879469    
gradMI       0.227061   0.098384   2.308 0.023350 *  
fathcoll     0.072219   0.077485   0.932 0.353863    
mothcoll    -0.083978   0.080836  -1.039 0.301707    
---
Signif. codes:  0 '***' 0.001 '**' 0.01 '*' 0.05 '.' 0.1 ' ' 1

Residual standard error: 0.3301 on 88 degrees of freedom
Multiple R-squared:  0.4026,    Adjusted R-squared:  0.2261 
F-statistic: 2.281 on 26 and 88 DF,  p-value: 0.002317
\end{verbatim}

\begin{itemize}
\tightlist
\item
  예측값 계산
\end{itemize}

\begin{Shaded}
\begin{Highlighting}[]
\NormalTok{predic\_gpa }\OtherTok{\textless{}{-}} \FunctionTok{predict}\NormalTok{(lm\_gpa,}\AttributeTok{newdata =}\NormalTok{ test)}

\NormalTok{result }\OtherTok{\textless{}{-}} \FunctionTok{data.frame}\NormalTok{(}
  \AttributeTok{actual =}\NormalTok{ test}\SpecialCharTok{$}\NormalTok{colGPA,}
  \AttributeTok{predict =}\NormalTok{ predic\_gpa}
\NormalTok{)}
\FunctionTok{head}\NormalTok{(result)}
\end{Highlighting}
\end{Shaded}

\begin{verbatim}
   actual  predict
1     3.0 2.581091
9     2.7 2.757334
10    3.8 3.799901
15    3.3 3.482618
16    2.6 2.850664
22    2.7 2.306191
\end{verbatim}

\begin{itemize}
\tightlist
\item
  예측값 시각화
\end{itemize}

\begin{Shaded}
\begin{Highlighting}[]
\NormalTok{result }\SpecialCharTok{|\textgreater{}} 
  \FunctionTok{ggplot}\NormalTok{(}\AttributeTok{mapping =} \FunctionTok{aes}\NormalTok{(}\AttributeTok{x =}\NormalTok{ actual, }\AttributeTok{y =}\NormalTok{ predict)) }\SpecialCharTok{+} 
  \FunctionTok{geom\_point}\NormalTok{() }\SpecialCharTok{+} 
  \FunctionTok{geom\_abline}\NormalTok{(}\AttributeTok{intercept =} \DecValTok{0}\NormalTok{ , }\AttributeTok{slope =} \DecValTok{1}\NormalTok{, }\AttributeTok{color =} \StringTok{"red"}\NormalTok{, }\AttributeTok{linetype =} \StringTok{"dashed"}\NormalTok{)}
\end{Highlighting}
\end{Shaded}

\pandocbounded{\includegraphics[keepaspectratio]{exer_files/figure-pdf/unnamed-chunk-6-1.pdf}}

\begin{Shaded}
\begin{Highlighting}[]
  \FunctionTok{labs}\NormalTok{( }
    \AttributeTok{x =} \StringTok{"실제 GPA"}\NormalTok{, }\AttributeTok{y =} \StringTok{"예측 GPA"}
\NormalTok{    )}
\end{Highlighting}
\end{Shaded}

\begin{verbatim}
$x
[1] "실제 GPA"

$y
[1] "예측 GPA"

attr(,"class")
[1] "labels"
\end{verbatim}

\begin{itemize}
\tightlist
\item
  MSE 와 RMSE
\end{itemize}

\begin{Shaded}
\begin{Highlighting}[]
\CommentTok{\# 계산}
\NormalTok{mse }\OtherTok{=} \FunctionTok{mean}\NormalTok{(result}\SpecialCharTok{$}\NormalTok{actual }\SpecialCharTok{{-}}\NormalTok{ result}\SpecialCharTok{$}\NormalTok{predict)}\SpecialCharTok{\^{}}\DecValTok{2}
\NormalTok{rmse }\OtherTok{=} \FunctionTok{sqrt}\NormalTok{(mse)}

\CommentTok{\# 값 확인 }
\FunctionTok{print}\NormalTok{(}\FunctionTok{paste}\NormalTok{(}\StringTok{"MSE = "}\NormalTok{, mse))}
\end{Highlighting}
\end{Shaded}

\begin{verbatim}
[1] "MSE =  0.000710112241020869"
\end{verbatim}

\begin{Shaded}
\begin{Highlighting}[]
\FunctionTok{print}\NormalTok{(}\FunctionTok{paste}\NormalTok{(}\StringTok{"RMSE = "}\NormalTok{, rmse))}
\end{Highlighting}
\end{Shaded}

\begin{verbatim}
[1] "RMSE =  0.0266479312709424"
\end{verbatim}

\subsubsection{1-4. 모델 선택}\label{uxbaa8uxb378-uxc120uxd0dd}

\begin{itemize}
\tightlist
\item
  AIC 방법을 활용한 모델 선택
\end{itemize}




\end{document}
